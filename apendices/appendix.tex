\chapter{Type Description}
\label{ap:signals}

\begin{table}[H]
\centering
\label{table:psrtype}
\begin{tabular}{l l p{55mm}}
\textbf{Signal} & \textbf{Type}      & \textbf{Description}                                         	\\ \hline \hline
cache\_line 	& std\_logic\_vector & the entire cach line                                        	\\ \hline
base\_addr  	& std\_logic\_vector & base physical address of the line                            	\\ \hline
valid       	& std\_logic         & asserted if the line is valid                                	\\ \hline
wr\_ptag    	& std\_logic         & asertedif the calculated ptag will be  writen in th e PTAG mem    \\  \hline
\end{tabular}                                                                                                                                   
\caption{\textbf{ptag\_sec\_req\_type} content description}                                                                                              
\end{table}                                                                                                                                     


\begin{table}[H]
\centering
\label{table:psvtype}
\begin{tabular}{l l p{55mm}}
\textbf{Signal} & \textbf{Type}      & \textbf{Description}                        \\ \hline \hline
ptag 		& std\_logic\_vector & the calculated ptag                         \\ \hline
valid 		& std\_logic         & asserted when  ptag is valid                \\ \hline
line\_secure 	& std\_logic         & asserted if the provided line is secure     \\ \hline
ready 		& std\_logic         & asserted when ready to receive a new line   \\  \hline
\end{tabular}                                                                                                                                   
\caption{\textbf{ptag\_sec\_val\_type} content description}                                                                                              
\end{table}    




\begin{table}[H]
\centering
\label{table:pmrqtype}
\begin{tabular}{l l p{55mm}}
\textbf{Signal} & \textbf{Type}      & \textbf{Description}                     \\ \hline \hline
we      & std\_logic;                 & active high write enable                \\ \hline
address & std\_logic\_vector           & memory address                         \\ \hline
data    & std\_logic\_vector           & data to be writen                      \\ \hline
\end{tabular}                                                                                                                                   
\caption{\textbf{ptag\_mreq\_type} content description}                                                                                              
\end{table}     

\begin{table}[H]
\centering
\label{table:pmrtype}
\begin{tabular}{l l p{55mm}}
\textbf{Signal} & \textbf{Type}      & \textbf{Description}                     \\ \hline \hline
data 		& std\_logic\_vector   & PTAG read from memory                  \\ \hline
 \end{tabular}                                                                                                                                   
\caption{\textbf{ptag\_mresp\_type} content description}                                                                                              
\end{table}    




\begin{table}[H]
\centering
\label{table:abhitype}
\begin{tabular}{l l p{55mm}}
\textbf{Signal} & \textbf{Type}      	& \textbf{Description}                  \\ \hline \hline
    hgrant	& std\_logic\_vector 	& bus grant                             \\ \hline
    hready	& std\_ulogic       	& transfer done                         \\ \hline
    hresp	& std\_logic\_vector	& response type                         \\ \hline
    hrdata	& std\_logic\_vector	& read data bus                         \\  \hline
    hcache	& std\_ulogic       	& cacheable                             \\ \hline
    hirq  	& std\_logic\_vector	& interrupt result bus                  \\ \hline
    testen	& std\_ulogic       	& scan test enable                      \\ \hline
    testrst	& std\_ulogic           & scan test reset          		\\  \hline
    scanen 	& std\_ulogic           & scan enable              		\\ \hline
    testoen 	& std\_ulogic           & test output enable      	 	\\ \hline
 \end{tabular}                                                                                                                         
\caption{\textbf{ahb\_mst\_in\_type} content description}                                                                                      
\end{table}                                                                                             

  
  
  
\begin{table}[H]
\centering
\label{table:ahbotype}
\begin{tabular}{l l p{55mm}}
\textbf{Signal} & \textbf{Type}      & \textbf{Description}                     \\ \hline \hline  

    hbusreq	& std\_ulogic           & bus request                           \\ \hline
    hlock	& std\_ulogic           & lock request                          \\ \hline
    htrans	& std\_logic\_vector	& transfer type                          \\  \hline
    haddr	& std\_logic\_vector 	& address bus (byte)                    \\ \hline
    hwrite	& std\_ulogic           & read/write                            \\ \hline
    hsize	& std\_logic\_vector	& transfer size                         \\ \hline
    hburst	& std\_logic\_vector	& burst type                             \\  \hline
    hprot	& std\_logic\_vector	& protection control                     \\ \hline
    hwdata	& std\_logic\_vector 	& write data bus                         \\ \hline
    hirq   	& std\_logic\_vector	& interrupt bus                          \\ \hline
    hconfig 	& ahb\_config\_type	& memory access reg.                    \\  \hline
    hindex  	& integer  		& diagnostic use only                   \\ \hline
 \end{tabular}                                                                                                                                 
\caption{\textbf{ahb\_mst\_out\_type} content description}                                                                                            
\end{table}                                                                                               


