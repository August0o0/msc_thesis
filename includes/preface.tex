% Este comando deve ficar aqui:
\paginasiniciais


\prefacesection{Dedicatória}
Lorem ipsum dolor sit amet, consectetur adipiscing elit. Phasellus vitae iaculis erat. Aliquam tristique consectetur ante, quis commodo lacus egestas in. Nullam semper elit nec eros pretium 


% Se houver epígrafe, descomente e edite:
% \begin{epigrafe}
% {\it
% Vita brevis,\\
% ars longa,\\
% occasio praeceps,\\
% experimentum periculosum,\\
% iudicium difficile.}
%
% \hfill (Hippocrates)
% \end{epigrafe}


% Agradecimentos ou Acknowledgements ou Agradecimientos
% \prefacesection{Agradecimentos}
% Os agradecimentos devem ocupar uma única página.


% Sempre deve haver um resumo em português:
\begin{resumo}
Escrever o resumo emportugues
\end{resumo}


% Sempre deve haver um abstract:
\begin{abstract}
% The abstract must have at most 500 words and must fit in a single page.

Based on Gaisler's \leon~\cite{Leon} \fpga~implementation, this work presents a proof-of-concept of \cshia. The main goal of our implementation was to improve the original version of the architecture and add more flexible design choices. Besides it is also presents an in depth description of the hardware implementation, the design tradeoffs and the integration between the architecture and a real processor.

\end{abstract}




% A lista de figuras é opcional:
\listoffigures

% A lista de tabelas é opcional:
\listoftables

% A lista de abreviações e siglas é opcional:
% \prefacesection{Lista de Abreviações e Siglas}

% A lista de símbolos é opcional:
% \prefacesection{Lista de Símbolos}

% Quem usa o pacote nomencl pode incluir:
%\renewcommand{\nomname}{Lista de Abreviações e Siglas}
%\printnomenclature[3cm]


% O sumário vem aqui:
\tableofcontents


% E esta linha deve ficar bem aqui:
\fimdaspaginasiniciais
