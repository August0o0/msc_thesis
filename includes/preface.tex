% Este comando deve ficar aqui:
\paginasiniciais


\prefacesection{Dedicatória}
Lorem ipsum dolor sit amet, consectetur adipiscing elit. Phasellus vitae iaculis erat. Aliquam tristique consectetur ante, quis commodo lacus egestas in. Nullam semper elit nec eros pretium 


% Se houver epígrafe, descomente e edite:
% \begin{epigrafe}
% {\it
% Vita brevis,\\
% ars longa,\\
% occasio praeceps,\\
% experimentum periculosum,\\
% iudicium difficile.}
%
% \hfill (Hippocrates)
% \end{epigrafe}


% Agradecimentos ou Acknowledgements ou Agradecimientos
% \prefacesection{Agradecimentos}
% Os agradecimentos devem ocupar uma única página.


% Sempre deve haver um resumo em português:
\begin{resumo}
Escrever o resumo emportugues
\end{resumo}


% Sempre deve haver um abstract:
\begin{abstract}


Standard design techniques to secure code execution are based on well-known cryptographic mechanisms and (micro) architecture features to encode bus transactions, or isolate secure code into trusted platforms, among others. 
Although such techniques usually provide proper levels of security, most of them are either inefficient, considerably impact processor (micro) architecture design, require extensive changes in the programming tool-chain, or are so complicated that may create unexpected security loopholes.
Aiming to address this security issues in code execution the Computer Security by Hardware-Intrinsic Authentication (\cshia) was proposed to provide authenticity by authenticating all memory blocks of an external memory using a unique key extracted from Physical Unclonable Functions (\pufs). 
Based on Gaisler's \leon~ \fpga~implementation, this work presents a proof-of-concept of \cshia, presenting the details and an in-depth description of the hardware implementation, the design tradeoffs, and the integration between the architecture and a real processor.  We show the FPGA resources, a performance evaluation with industry standard benchmarks and power and area estimations. \mario{É comum finalizar o abstract com uma frase resaltando as vantagens e contribuições }

\end{abstract}




% A lista de figuras é opcional:
\listoffigures

% A lista de tabelas é opcional:
\listoftables

% A lista de abreviações e siglas é opcional:
% \prefacesection{Lista de Abreviações e Siglas}

% A lista de símbolos é opcional:
% \prefacesection{Lista de Símbolos}

% Quem usa o pacote nomencl pode incluir:
%\renewcommand{\nomname}{Lista de Abreviações e Siglas}
%\printnomenclature[3cm]


% O sumário vem aqui:
\tableofcontents


% E esta linha deve ficar bem aqui:
\fimdaspaginasiniciais
