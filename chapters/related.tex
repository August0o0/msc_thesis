
\soc~devices have a number of features which distinguish them from other traditional electronic solutions. Their dedicated nature allows the adoption of more intrusive protection, while posing challenging energy and performance requirements. Unfortunately, traditional security solutions based on typical cryptographic mechanisms (e.g. bus encryption)  can have a significant impact in device cost, energy efficiency and performance. One way to go around that is to consider approaches which enable a deep integration of device hardware-intrinsic features and program execution, as those offered by \pufs.

Qualitative analyses of \pufs~have already been done in the literature~\cite{Katzenbeisser2012} motivated by several applications such as cryptographic key generation~\cite{Suh2007, Bhargava2014} and true random number generation~\cite{Leest2012, Herrewege2013}. Unlike those works, which aim at evaluating the quality of a standalone \puf-inspired mechanism, this work focus on proposing and analyzing a \puf-based micro-architecture mechanism to enable secure code execution. %\new{On big conundrum on using \pufs~is how to prevent unintended bit-flips on responses used to compound the cryptographic key. An approach to deal with that is to use post-processing schemes like Fuzzy Extractors \cite{}. Despite Fuzzy Extractors be a simple and known circuits, they the weakest spot on the architecture and can leak the cryptographic key through side channel attacks. }

Most of the preliminary work on secure code execution aimed at keeping instructions and data secure from scrutiny, by using mechanisms like bus encryption. In~\cite{Elbaz2005}, Elbaz \etal~performed a comprehensive survey of bus encryption, where they describe many possible ways of using cryptographic algorithms in \soc~architectures, so as to ensure that no malicious instruction\slash{}data would be executed by the CPU. The major shortcoming of these solutions is the usage of on-chip secret key storage in non-volatile memories which enable off-line key recovery attacks~\cite{towardshardwaresecurity2010}.

AEGIS, the secure processor proposed by Suh \etal~in~\cite{Suh2005}, employs \pufs~as a cryptography primitive to uniquely authenticate code and data in order to prevent both software and physical attacks. They present a toolchain for developing secure software for their architecture which includes a secure operating system to manage different levels of memory protection. Although the presented toolchain does not require modifications in the processor architecture, it demands extensive changes in the \soc~architecture, in addition to changes in the compiler and operating system. Moreover, AEGIS \emph{does not} ensure full-time security from power-on to power-off; i.e. the system runs unprotected until the security kernel loads the system. In addition, physical attacks were neither evaluated nor simulated. Different circuits used in AEGIS, like \pufs~and post-processing schemes for key extraction such as Fuzzy Extractors, have been successfully attacked with side-channel \cite{Merli2011,Tajik2016:Photonic} and semi-invasive attacks \cite{Tajik2015:LaserAttack}. While semi-invasive attacks are hard to repeal, side-channel attacks have few known countermeasures \cite{Merli2013:Masking} that can be easily adopted.

