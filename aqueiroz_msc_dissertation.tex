% Escolha: Portugues ou Ingles ou Espanhol.
% Para a versão final do texto, após a defesa acrescente Final:


\documentclass[Ingles,Final]{ic-tese-v3}
% \documentclass[Portugues,Final]{ic-tese-v3}

\usepackage[latin1,utf8]{inputenc}

% \usepackage{pdflscape}
\usepackage{rotating}
\usepackage{tikz}


% Para acrescentar comentários ao PDF descomente:
\usepackage
 [pdfauthor={Augusto F. R. Queiroz},
  pdftitle={Secure code execution using PUF authentication},
  pdfkeywords={PUF}]%,
%   pdfproducer={Latex with hyperref},
%   pdfcreator={pdflatex}]
{hyperref}


% *** GRAPHICS RELATED PACKAGES ***w

\usepackage{graphicx,url, float}
%    \ifCLASSINFOpdf
%        \usepackage[pdftex]{graphicx,url}
        \graphicspath{{figures/pdf/}}
        \DeclareGraphicsExtensions{.pdf}
%    \else
%        \usepackage[dvips]{graphicx,url}
%        \graphicspath{{figures_eps/}}
%        \DeclareGraphicsExtensions{.eps}
%    \fi

% *** MATH PACKAGES ***
\usepackage{amssymb}  

% *** ALGORITHM PACKAGES ***
\usepackage{algorithm}
\usepackage{algorithmic}

% *** CITATION PACKAGE ***
\usepackage{cite}

% *** MISC UTILITY PACKAGES ***
% \usepackage{ulem}
\usepackage{multirow}
\usepackage[printonlyused]{acronym}

% *** COMMANDS ***
\newcommand{\cald}{{\cal D}}
\newcommand{\calh}{{\cal H}}
\newcommand{\calj}{{\cal J}}
\newcommand{\calm}{{\cal M}}
\newcommand{\caln}{{\cal N}}
\newcommand{\cals}{{\cal S}}
\newcommand{\calt}{{\cal T}}
\newcommand{\calv}{{\cal V}}

\begin{document}

% Escolha entre autor ou autora:
\autor{Augusto Fernandes Ribas Queiroz }


% Sempre deve haver um título em português:
\titulo{Execução segura de códigos utilizando PUF para autenticação}

% Se a língua for o inglês ou o espanhol defina:
\title{Secure code execution using PUF authentication}

% Escolha entre orientador ou orientadora. Inclua os títulos acadêmicos:
\orientador{Prof. Dr.Guido Costa Souza de Araújo}
%\orientadora{Profa. Dra. Nome da Orientadora}

% Escolha entre coorientador ou coorientadora, se houver.  Inclua os títulos acadêmicos:
\coorientador{Prof. Dr.Mário Lúcio Côrtes}


% Escolha entre mestrado ou doutorado:
\mestrado
%\doutorado

% Se houve cotutela, defina:
%\cotutela{Universidade Nova de Plutão}

\datadadefesa{22}{03}{2019}

% Para a versão final defina:
\avaliadorA{Prof. Dr. someone}{Universidade Estadual de Campinas}
\avaliadorB{Prof. Dr.Someonel}{Universidade Federal de Santa Catarina}
\avaliadorC{Profa. Dra. someone}{Universidade Estadual de Campinas}


% Para incluir a ficha catalográfica em PDF na versão final, descomente e ajuste:
%\fichacatalografica{includes/Ficha-Catalografica-Protocolo-325698704.pdf}


% Este comando deve ficar aqui:
\paginasiniciais


\prefacesection{Dedicatória}
Lorem ipsum dolor sit amet, consectetur adipiscing elit. Phasellus vitae iaculis erat. Aliquam tristique consectetur ante, quis commodo lacus egestas in. Nullam semper elit nec eros pretium 


% Se houver epígrafe, descomente e edite:
% \begin{epigrafe}
% {\it
% Vita brevis,\\
% ars longa,\\
% occasio praeceps,\\
% experimentum periculosum,\\
% iudicium difficile.}
%
% \hfill (Hippocrates)
% \end{epigrafe}


% Agradecimentos ou Acknowledgements ou Agradecimientos
% \prefacesection{Agradecimentos}
% Os agradecimentos devem ocupar uma única página.


% Sempre deve haver um resumo em português:
\begin{resumo}
Lorem ipsum dolor sit amet, consectetur adipiscing elit. Phasellus vitae iaculis erat. Aliquam tristique consectetur ante, quis commodo lacus egestas in. Nullam semper elit nec eros pretium dapibus. Ut eget porta metus. Mauris rhoncus vel magna non faucibus. Ut a ornare elit. Morbi sagittis quam nec risus laoreet, tincidunt volutpat ex venenatis. Sed ultrices felis quis felis scelerisque gravida a non neque. Etiam sed nisi neque. Ut lobortis pulvinar facilisis. Vestibulum quis enim bibendum, iaculis nulla non, consequat nunc. Suspendisse sed hendrerit tortor, non maximus lorem. Ut sit amet turpis eget libero condimentum scelerisque. Quisque sed dolor metus.

Cras ac tincidunt tellus. Morbi id interdum magna, a ornare justo. Praesent tincidunt tempus porta. Sed iaculis fermentum nibh non porta. Aenean mattis sapien purus, ut ornare risus ornare non. Integer sit amet neque auctor, interdum mi in, sodales sem. Pellentesque at rutrum quam, eget posuere elit. Donec nec tortor egestas, laoreet ligula sit amet, luctus felis. Proin sit amet tempor odio. Sed orci orci, gravida dignissim molestie id, fringilla nec ligula.

Aenean pharetra, massa eu dapibus malesuada, lacus nisl auctor magna, in tempor nulla ante et ante. Mauris in purus lacus. Nam ac enim et ipsum condimentum vehicula. Mauris pulvinar facilisis lacus, non ultrices risus tempus a. Lorem ipsum dolor sit amet, consectetur adipiscing elit. Donec auctor sagittis lacinia. Fusce porta dapibus vulputate.

Nam posuere lacus nulla, nec pharetra nisi tempor vitae. Sed ac erat eget lacus ultrices tincidunt. Aliquam egestas quam vitae magna semper tempor. Nam at diam sit amet eros bibendum efficitur at ac neque. Morbi rhoncus ullamcorper erat, porta rutrum lacus tincidunt non. Nunc eget tellus eu nulla ornare ullamcorper et sit amet nulla. Vivamus in erat ultrices, faucibus velit sit amet, feugiat massa. Ut lacinia mattis quam ac posuere. Praesent et eros mauris. Nullam malesuada mauris diam, sed varius tortor posuere at. Aenean volutpat suscipit tortor vel blandit. Nullam euismod est a suscipit tempus.
\end{resumo}


% Sempre deve haver um abstract:
\begin{abstract}
% The abstract must have at most 500 words and must fit in a single page.

Lorem ipsum dolor sit amet, consectetur adipiscing elit. Phasellus vitae iaculis erat. Aliquam tristique consectetur ante, quis commodo lacus egestas in. Nullam semper elit nec eros pretium dapibus. Ut eget porta metus. Mauris rhoncus vel magna non faucibus. Ut a ornare elit. Morbi sagittis quam nec risus laoreet, tincidunt volutpat ex venenatis. Sed ultrices felis quis felis scelerisque gravida a non neque. Etiam sed nisi neque. Ut lobortis pulvinar facilisis. Vestibulum quis enim bibendum, iaculis nulla non, consequat nunc. Suspendisse sed hendrerit tortor, non maximus lorem. Ut sit amet turpis eget libero condimentum scelerisque. Quisque sed dolor metus.

Cras ac tincidunt tellus. Morbi id interdum magna, a ornare justo. Praesent tincidunt tempus porta. Sed iaculis fermentum nibh non porta. Aenean mattis sapien purus, ut ornare risus ornare non. Integer sit amet neque auctor, interdum mi in, sodales sem. Pellentesque at rutrum quam, eget posuere elit. Donec nec tortor egestas, laoreet ligula sit amet, luctus felis. Proin sit amet tempor odio. Sed orci orci, gravida dignissim molestie id, fringilla nec ligula.

Aenean pharetra, massa eu dapibus malesuada, lacus nisl auctor magna, in tempor nulla ante et ante. Mauris in purus lacus. Nam ac enim et ipsum condimentum vehicula. Mauris pulvinar facilisis lacus, non ultrices risus tempus a. Lorem ipsum dolor sit amet, consectetur adipiscing elit. Donec auctor sagittis lacinia. Fusce porta dapibus vulputate.

Nam posuere lacus nulla, nec pharetra nisi tempor vitae. Sed ac erat eget lacus ultrices tincidunt. Aliquam egestas quam vitae magna semper tempor. Nam at diam sit amet eros bibendum efficitur at ac neque. Morbi rhoncus ullamcorper erat, porta rutrum lacus tincidunt non. Nunc eget tellus eu nulla ornare ullamcorper et sit amet nulla. Vivamus in erat ultrices, faucibus velit sit amet, feugiat massa. Ut lacinia mattis quam ac posuere. Praesent et eros mauris. Nullam malesuada mauris diam, sed varius tortor posuere at. Aenean volutpat suscipit tortor vel blandit. Nullam euismod est a suscipit tempus.

\end{abstract}




% A lista de figuras é opcional:
\listoffigures

% A lista de tabelas é opcional:
\listoftables

% A lista de abreviações e siglas é opcional:
% \prefacesection{Lista de Abreviações e Siglas}

% A lista de símbolos é opcional:
% \prefacesection{Lista de Símbolos}

% Quem usa o pacote nomencl pode incluir:
%\renewcommand{\nomname}{Lista de Abreviações e Siglas}
%\printnomenclature[3cm]


% O sumário vem aqui:
\tableofcontents


% E esta linha deve ficar bem aqui:
\fimdaspaginasiniciais


% O corpo da dissertação ou tese começa aqui:
%\include{chapters/introduction}
\chapter{Introduction}
\label{chap:introduction}
	\section{Contributions}
	\label{sec:contributions}
	\section{Publications}
	\label{sec:publications}
	\section{Organization of the dissertation}
	\label{sec:organization_of_dissertation}

\chapter{Related Work}
\label{chap:related_work}

\chapter{Fundamental concepts}
\label{chap:fundamental_concepts}  

\section{concept1}
\label{sec:concept1}

\chapter{CSHIA  Architecture}
\label{chap:cshia_architecture}

\chapter{CSHIA  Prototype}
\label{caho:cshia_prototype}


\chapter{CSHIA  Evaluation}
\label{chap:cshia:evaluation}

\chapter{Conclusion and Future Work}
\label{chap:conclusion}
    \section{Conclusion}
    \label{sec:conclusion}
\section{Future Work}
    \label{sec:future_work}

% As referências:
\bibliographystyle{plain}
\bibliography{aqueiroz_msc_dissertation}


% Os anexos, se houver, vêm depois das referências:
%\appendix
%\chapter{Illustrative example files}
\label{appendix:example_files}

This appendix contains the plain text files that correspond to the
application, cloud, virtual machine repository, and generated
schedule of the illustrative example presented in subsection
\ref{sec:illustrative_example}.

\section{Application}


    \begin{tabular}{l c c l}
        $n$ & : & 4 &\\
        $I$ & : & [ & (1) 1 3 3 1]\\

        $S$ & : & [ & (1) 2 1 2 1]\\
        $B$ & : & [ & (1 1) 0 2 2 0 \\
            &   &   & (2 1) 0 0 0 1 \\
            &   &   & (3 1) 0 0 0 1 \\
            &   &   & (4 1) 0 0 0 0\\
            &   & ] & \\
        $D$ & : & [ & (1 1) 0 1 1 0 \\
            &   &   & (2 1) 0 0 0 1 \\
            &   &   & (3 1) 0 0 0 1 \\
            &   &   & (4 1) 0 0 0 0\\
            &   & ] & \\
    \end{tabular}



\section{Grid/Private Cloud}
    \begin{tabular}{l c c l}
        $m$  & : & 4 & \\       
        $TI$ & : & [ & (1) 1.000000 1.000000 1.000000 1.000000]\\
        $C$  & : & [ & (1) 1 1 1 1]\\

        $TB$ & : & [ & (0 1) 0.000000 2.000000 2.000000 2.000000 \\
             &   &   & (1 1) 2.000000 0.000000 2.000000 2.000000\\
             &   &   & (2 1) 2.000000 2.000000 0.000000 2.000000\\
             &   &   & (3 1) 2.000000 2.000000 2.000000 0.000000\\
             &   & ] & \\
        $N$  & : & [ & (0 1) 1 1 1 1\\
             &   &   & (1 1) 1 1 1 1\\
             &   &   & (2 1) 1 1 1 1\\
             &   &   & (3 1) 1 1 1 1\\
             &   & ] & \\
        $TR$ & : & [ & (1) 1.000000 1.000000 1.000000 1.000000]\\
    \end{tabular}



\section{Virtual machines repository}

    \begin{tabular}{l c c l}
        $o$  & : & 4 & \\
        $SV$ & : & [ & (1) 1 2 3 4]\\
        $BV$ & : & [ & (1) 4.000000 4.000000 4.000000 4.000000]\\
        $TV$ & : & [ & (1) 2.000000 2.000000 2.000000 2.000000]\\
    \end{tabular}



\section{Resulting schedule}

$T_{max}=24$
%     \chapter{The grid application generator}
    \ldots



\end{document}
